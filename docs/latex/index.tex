G\+L\+AC is a program which generates gauge configurations and has the possibility of applying gradient flow on them as well.

It takes input in the form of .json files generated by create\+Jobs.\+py, a python2 command line interfacing tool. create\+Jobs.\+py will create relevant folders for the run. In the case that we are running on an H\+PC cluster using Slurm or Torque, it will directly initiate the run by submitting the jobs unless {\ttfamily --dryrun} is specified.

See \mbox{\hyperlink{page1}{Quickstart to G\+L\+AC}} for a short guide in how to use G\+L\+AC.

\begin{DoxySeeAlso}{See also}

\begin{DoxyItemize}
\item \href{https://github.com/hmvege/LQCDMasterThesis}{\texttt{ The M.\+Sc. thesis by Mathias M Vege which was the basis for this code.}}
\item \href{https://github.com/hmvege/LatViz}{\texttt{ Lat\+Viz\+: A lattice visulization program}} which can generate animations based on the output of the \mbox{\hyperlink{class_lattice_action_charge_density}{Lattice\+Action\+Charge\+Density}} observable. 
\end{DoxyItemize}
\end{DoxySeeAlso}
